\subsection{Der Beweisassistent Agda}
Agda ist ein Programm, in dem wir Terme und Typen in abhängiger Typentheorie eingeben und prüfen können, ob diese den Regeln entsprechen. Agda kann man hier online testen: \url{https://agdapad.quasicoherent.io/}. Und hier findet man eine Anleitung zur Installation: \url{https://agda.readthedocs.io/en/latest/getting-started/installation.html}. Für die meisten Linux-Distributionen sollte es einfach genügend aktuelle Pakete geben. Für Windows-Nutzer (und Leute die nicht in Emacs/Vi arbeiten möchten) ist es wahrscheinlich eine gute Idee, ``VS Code'' mit dem agda-mode plugin zu verwenden.

In der ersten Übung wir uns Agda angeschaut - mit diesen Aufgaben: \url{https://felix-cherubini.de/HoTT.agda}.
Hier gibt es noch eine sehr nette Art Agda besser kennen zu lernen von Ingo Blechschmidt:
\url{https://lets-play-agda.quasicoherent.io/}

Der Stand der Vorlesung, wie man ihn für Übungsblatt 2 braucht, ist in folgender Agda-Datei zusammengefasst: \url{https://felix-cherubini.de/blatt2.agda}.
