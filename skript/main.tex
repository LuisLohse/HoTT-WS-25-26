% latexmk -pdflatex='xelatex %O %S' -pvc -pdf main.tex
\documentclass{../util/hott}

\setenumerate[1]{label=(\alph*),topsep=0pt}
\setenumerate[2]{label=(\roman*),topsep=0pt}

\title{Vorlesungsskript zur Homotopietypentheorie}
\author{Felix Cherubini}

\makeindex

\begin{document}

\maketitle

\tableofcontents
 \pagebreak
Dieses Skript entsteht als Begleitmaterial zur Vorlesung ``Homotopietypentheorie'' (HoTT), die ich im Wintersemester 25/26 an der Universität Augsburg halte.
Zweck dieses Skripts ist es, den Zuhörern und mir selbst zur Erinnerung an die Vorlesungsinhalte zu dienen --
bei der Beschäftigung mit dem Thema ist es hilfreich in Lehrbücher zu schauen.
Das sogenannte \href{https://homotopytypetheory.org/book/}{``HoTT-Book''} ist sicher eine gute Quelle.

Wer Fehler findet und diese korrigieren oder darauf aufmerksam machen will, kann das auf der github-Seite dieses Skripts machen:
\href{https://github.com/felixwellen/HoTT-Vorlesung}{https://github.com/felixwellen/HoTT-Vorlesung}.
Dort gibt es auch die Möglichkeit, sogenannte ``Issues'' anzulegen.
Hier könnten sie etwa darüber informieren, wenn sie eine Passage nicht verstehen oder einen Fehler gefunden haben.
Sie haben über sogenannte ``Pull requests'' die Möglichkeit, Fehler auch selbst zu korrigieren.
Es gibt einen Vorläufer dieses Skripts vom Sommersemester 2021, an dessen Entstehung außer mir auch Daniel Albert und Lukas Stoll durch zahlreiche Korrekturen und Verbesserungen mitgewirkt haben.


Die Homotopietypentheorie ist eine eigenständige Art Mathematik zu betreiben und basiert auf einer abhängigen Typentheorie.
Das bringt mit sich, dass wir zunächst erlernen werden, wie man in Homotopietypentheorie Objekte konstruiert, Aussagen formuliert und Beweise führt.
Nach den Grundlagen werden wir uns in Richtung Homotopietheorie orientieren, einem Teilgebiet der Mathematik, in dem einige Vorzüge der Ho\-mo\-to\-pie\-ty\-pen\-theo\-rie zur Geltung kommen.

Im Folgenden werden wir nach und nach \begriff{Regeln} einführen (oder zumindest erwähnen), die schließlich zusammen eine Typentheorie ergeben.
Diese werden wir noch um das sogenannte \begriff{Univalenzaxiom} erweitern.

\printindex

\end{document}
