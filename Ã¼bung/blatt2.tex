% latexmk -pdf -pvc blatt2.tex
\documentclass{hott-übung}

\begin{document}

\setcounter{blattnummer}{2}
\setcounter{aufgabennummer}{0}

\blatt

\aufgabe{}
Analog zum Typ $\eins$ aus der Vorlesung gibt es auch einen induktiven Typ $\zwei$,
mit zwei Konstruktoren, die jeweils keine Argumente haben.
Finde passende Regeln für diesen Typ, insbesondere einen Eliminator.\\
\emph{Bemerkung:} Wir hatten diesen Typ in Agda als ``Bool'' kennen gelernt. 

\aufgabe{}
Zeige, dass die Addition $+$ auf $\N$ assoziativ ist:
Für alle $n,k,l:\N$ haben wir $(n+k)+l=_\N n+(k+l)$.
Ganz formal ausgedrückt ist die Aufgabe also, einen Term des Typs
\[
  (n,k,l:\N) \to (n+k)+l=_\N n+(k+l)
\]
zu konstruieren.

\aufgabe{}
Sei $A$ ein Typ. Zeige:
\begin{enumerate}[(a)]
\item Für alle $ p:x=y$ haben wir $ p\kon p^{-1}=\refl_x= p^{-1}\kon p$.
\item Für alle $ p:x=_Ay$, $ q:y=_Az$, $ r:z=_Au$ gilt
   $( p\kon q)\kon r= p\kon( q\kon r)$.
\end{enumerate}

\aufgabe{}
Konstruiere Abbildungen $\varphi:\eins\sqcup \eins\to \zwei$ und $\psi:\zwei\to \eins\sqcup \eins$
zusammen mit Termen in
\[
  (x:\eins\sqcup \eins)\to \psi (\varphi(x))=x\quad \text{ und }\quad (x:\zwei)\to \varphi(\psi(x))=x
\]

\end{document}
